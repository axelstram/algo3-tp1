\subsection{Sobre la complejidad }
 En todos los ejercicios trabajamos sobre el Modelo Uniforme. Por lo tanto suponemos en todos los casos que las asignaciones y comparaciones 
 de tipos b�sicos, definici�n de variables, creaci�n de iteradores y operaciones matem�ticas est�ndar tienen una complejidad constante.
 
 \subsection{Sobre la documentaci�n }
  Usamos como documentaci�n la pagina http://en.cppreference.com/w/, y aparece en cada c�lculo de complejidad el link correspondiente
  a la funci�n 
  usada.
\subsection{Sobre la implementaci�n }
La carpeta del tp contiene una subcarpeta por ejercicio.

Dentro de estas se encuentra los archivos usados en la implementaci�n del problema, 
y otra carpeta, llamada Test, que contiene todo lo relativo al testeo del ejercicio. La carpeta de Test contiene el
generador de instancias aleatorias, 
los test de correctitud, o casos $borde$, un archivo test random usado para generar las mediciones y un archivo.xls
donde figuran las mismas,
 que fueron usadas para crear los gr�ficos.
 
Tambi�n se adjuntan Makefile's para asegurar que el proceso de compilaci�n sea el mismo. 
 
 \subsection{Sobre la ejecuci�n }
 Para la ejecuci�n de los ejecutables, en los tres casos, debe pasarse como par�metro un 1 o un 0 que indica si debe 
 tomar mediciones o no. 1 significa que si, 0 que no debe hacerlo. 
 En el caso particular del ejercicio 3, adem�s deben pasarse par�metros adicionales indicando que podas se desea aplicar 
 (m�s detalles en el ejercicio correspondiente). Finalmente, se debe usar el operador 
 de redirecci�n <, seguido del nombre del archivo que contiene las instancias de test.
 La ejecuci�n de los generadores de instancias es directa, luego se pide ingresar los par�metros por consola, durante la ejecuci�n.  
 
  
 \subsection{Sobre la experimentaci�n }
A fin de reducir el $ruido$, ejecutamos 50 veces la resoluci�n de la misma instancia, y tomamos el tiempo final como
el promedio de los tiempos 
sumados. El tiempo se mide en nanosegundos.
\subsubsection{Medici�n de Tiempos}
Para realizar la medici�n de tiempos, utilizamos el tipo timespec, que es una estructura con los siguientes tipos:
\newline
\begin{verbatim}
time_t  tv_sec   
long    tv_nsec 
\end{verbatim}
Que almacenan el tiempo en segundos y nanosegundos, respectivamente.
\newline
Para medir el tiempo antes y despu�s de correr el algoritmo, utilizamos la funci�n \textbf{clock\_gettime}, y luego
para obtener la diferencia entre el tiempo de inicio y el tiempo final, utilizamos la funci�n diff, definida a continuaci�n:

\begin{verbatim}
timespec diff(timespec start, timespec end) {
    timespec temp;
    if ((end.tv_nsec-start.tv_nsec)<0) {
        temp.tv_sec = end.tv_sec-start.tv_sec-1;
        temp.tv_nsec = 1000000000+end.tv_nsec-start.tv_nsec;
    } else {
        temp.tv_sec = end.tv_sec-start.tv_sec;
        temp.tv_nsec = end.tv_nsec-start.tv_nsec;
    }
    return temp;
}

\end{verbatim}

